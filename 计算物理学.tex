\documentclass{ctexart}
\usepackage{amsmath}
\usepackage{mathtools}
\title{计算物理学作业}
\author{艾鑫}

\newcounter{mycnt}
\setcounter{mycnt}{0}
\newenvironment{problem}[1][1.1]{\noindent \stepcounter{mycnt}\themycnt.(#1)}{

}

\newenvironment{answer}{\textbf{解}:}{
\vspace{0.5cm}
}

\newcommand\diff{\,\mathrm{d}}

\begin{document}
\maketitle

\begin{problem}[3.10]
用4点高斯求积公式编程计算积分
\begin{equation}
  I = \int_{1.4}^{2.0} \int_{1.0}^{1.5} \ln (x+2y) \diff x \diff y
\end{equation}
\end{problem}

\begin{answer}
  将积分区间$R = \{(x,y)|1.4 \leq x \leq 2.0, \, 1.0 \leq y \leq 1.5\}$变换到$R' = \{(u,v)|-1 \leq u \leq -1, \, -1 \leq v \leq 1\}$, 即有
  \begin{equation}
    \left\{
      \begin{lgathered}
        x = \frac{1}{2}(b_2 + a_2) + \frac{1}{2} (b_2 - a_2)u \\
        y = \frac{1}{2}(b_1 + a_1) + \frac{1}{2} (b_1 - a_1)v
      \end{lgathered}
    \right.
  \end{equation}
因此有
\begin{equation}
  I = 0.0755 \int_{-1}^{1} \int_{-1}^{1} \ln (0.3 u + 0.5 v + 4.2) \diff u \diff v
\end{equation}
使用$n=3$的4点高斯求积公式, 有
\end{answer}

\begin{problem}[4.7]
  应用龙格-库塔法求初值问题
  \begin{equation}
    \left\{
    \begin{lgathered}
      y'' + 2ty' + t^2 y = e^t, \quad 0 \leq t \leq 1 \\
      y(0) = 1, y'(0) = -1
    \end{lgathered}
    \right.
  \end{equation}
的数值解,取步长$h = 0.1$.
\end{problem}

\begin{answer}
  测试答案
\end{answer}

\begin{problem}[8.4]
  试采用有限差分法编程求解单位圆域中泊松方程在网格节点处的值
  \begin{equation}
    \left\{
      \begin{lgathered}
        \nabla^2u = -50r^2 \sin (2\phi) \\
        u(1, \phi) = 0
      \end{lgathered}
    \right.
  \end{equation}
  其中$u(1,\phi)=0$表示半径为1的单位圆边界上$u$值为$1$. 取$h = 0.1, \omega = 1.25, \varepsilon = 10^{-5}, M = 16$, 并与解析解$u = \frac{25}{6}r^2(1-r^2) \sin(2\phi)$作比较.
\end{problem}

\begin{answer}
测试回答
\end{answer}


\begin{problem}[11.1]
  编程计算例11.1中点3的电势和1,2两点的电量(见图11.9).
\end{problem}

\begin{answer}
测试回答
\end{answer}

\begin{problem}[12.9]
  采用蒙特卡罗方法编程求解一下方程.
  \begin{enumerate}
  \item[(1)] $e^{-x^3}-\tan x + 800 = 0 \,(0 < x < \pi / 2)$;
  \item[(2)] $x + 5e^{-x} - 5 = 0 \, (0 < x < 10)$.
  \end{enumerate}
\end{problem}

\begin{answer}
测试回答
\end{answer}

\end{document}